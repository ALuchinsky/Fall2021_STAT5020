\section{Introduction}
The most basic definition of life expectancy is the probable number of years of life a person in a particular cohort (people born in the same year and same country) will live. Life expectancy is used widely for determining life insurance rates, pension calculations, assessing the health of a population, etc. Life expectancy differs in a particular year depending on the age of the cohort being studied, for example, the life expectancy for a child born this year in a particular country would be different than the life expectancy for a 50-year-old man born in the same country. According to “Our World in Data,” life expectancy has increased over the last 200 years due to improvements in health, with many countries doubling the average life expectancy. 
The World Health Organization (WHO) compiled data over multiple years for 193 countries in the world related to health data. The data was combined with data from the United Nations data to analyze factors that impact life expectancy. The specific factors will be discussed in later sections, but the factors fit into the following broad categories: mortality factors, economic factors, immunization factors, and social factors. 

\subsection{Objectives}
\label{objectives}


\begin{enumerate}
\item To create a regression model for life expectancy based on the factors considered in the WHO data.
\item To determine which factors are most correlated with life expectancy.
\item To determine how changes in factors will impact life expectancy.   
\end{enumerate}


