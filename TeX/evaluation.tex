\section{Evaluation}
\label{sec:evaluation}

\subsection{Multicollinearity}
\label{sec:multicollinearity}
Multicollinearity among the regressor variables was checked and those with high variable inflation factor or VIF were removed. The VIF of a variable is the ratio of the variance of the overall model to the variance of a model with only that variable. A VIF greater than 4 indicates moderate multicollinearity and a VIF greater than 10 indicates severe multicollinearity. The regressor variables of Hepatitis.B, transformed logGDP, logPopulation, thinness.5.9.years, Income.composition.of.resources, and, Schooling were removed because these had greater than 4 VIF, and their p-value in the model summary indicated lower significance. 
After removing these variables, the model was fitted and again the VIF and the Pearson correlation coefficients for the variables were calculated. A Pearson correlation coefficient is a value that calculates the strength of a linear relationship between the two variables. The Pearson correlation coefficient can take a value between -1 and 1, with 1 and -1 denoting perfect linearity and 0 denoting no linearity.

\subsection{Identification of Influential Points}
\label{sec:influential-points}
Further, the points that play an important role than others in determining the regression line were found. High leverage points were looked for by finding fitted values that were greater than 2* (k+1)/n, where k is the number of variables in the model and n is the number of observations used to create the model. The outliers were searched by identifying where the absolute internally studentized and externally studentized residuals are greater than 3. Then, three methods were used to identify potentially influential points. The first method that was used is Difference in Fits, or DFFITS, which is a measure of the difference in fitted values when a chosen observation is removed from the model. A point was considered influential if the absolute value of $DFFITS_i$ was greater than $2* \sqrt{(k+2)/(n-k-2)}$. The second method that was used was Cook’s distance, $D_i$, which identifies influential points by determining which observations had high leverage or were outliers. An observation was considered possibly influential if the value returned from the R function: \textit{cooks.distance()} was greater than 1. The third method used was COVRATIO. An observation was considered possibly influential if the value of $COVRATIO_i$ was greater than $1+ 3*((k+1)/n)$, or less than $1- 3*((k+1)/n)$.

Ten such observations overlapping in influential, high leverage, and outliers were removed. 

\subsection{Residual Shapiro Test}
\label{sec:residual-shapiro-test}
The Shapiro-Wilk test was conducted to check for the normal distribution of the error terms where the null hypothesis is that the errors follow a normal distribution. As shown in the figure \ref{fig:shapiro}The p-value obtained from the test was 0.3442788 which is greater than the \textit{alpha} of 0.05. Hence, we failed to reject the null hypothesis and thus it was found that the error terms follow the normal distribution. 

\begin{figure}
  \centering
  \includegraphics[width = 0.9\textwidth]{figures/Shapiro.PNG}
  \caption{Plot for error terms with p-value from Shapiro-Wilk test}
  \label{fig:shapiro}
\end{figure}


% ## assessing model assumptions
